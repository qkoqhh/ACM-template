%\documentclass[12pt,a4paper]{article}
\usepackage{xeCJK}
\usepackage{fancyhdr}%页眉页脚
\usepackage{listings}%代码块
\usepackage[margin=1in]{geometry}%页边距
\usepackage{graphics}%图片
\usepackage{fontspec}
\usepackage{amsmath}
\usepackage{amsfonts}
\usepackage{eqnarray}
\usepackage{tabularx}
%定义页眉页脚
\pagestyle{fancy}
\fancyhf{}
\fancyhead[C]{\selectfont \textsc{ACM Template}}
\fancyhead[L]{\rightmark}
\fancyhead[R]{Geometry Rhythm}
%\lfoot{lfoot}
%\cfoot{qkoqhh}
\rfoot{\thepage}
%\renewcommand{\footrulewidth}{0.5pt}

%\rmfamily%罗马字体




\lstset{
	language=C,%代码的语言
	numbers=left,%行号
	frame=single,%
	basicstyle=\footnotesize,
	extendedchars=false,
	basicstyle=\small\ttfamily,
	%	tabsize=2,
	breaklines=true,
	showstringspaces=false
}



\title{ACM Template}
\author{qkoqhh}
%\begin{document}
	\newpage
	\section{通用技术}
	\subsection{Head}
	\lstinputlisting{./source/head.cpp}
	\newpage
	\subsection{Vimrc}
	\lstinputlisting[language=clean]{./source/vimrc}
	\newpage
	\subsection{对拍}
	\subsubsection*{Linux}
	\lstinputlisting[language=clean]{./source/cmp.sh}
	\vspace{2cm}
	\subsubsection*{Windows}
	\lstinputlisting[language=clean]{./source/cmp.bat}
	\newpage
	\subsection{调试技巧}
	主要从大大的 debug.pdf 里面摘了一些自己不熟悉的东西
	\subsubsection{栈空间}
	\paragraph{为本机分配尽量大的栈空间}~\\
	\begin{lstlisting}[numbers=none]
		ulimit -s unlimited
	\end{lstlisting}
	\paragraph{将栈空间调整为固定大小(单位:KB)}~\\
	\begin{lstlisting}[numbers=none]
		ulimit -s 131072
	\end{lstlisting}
	\paragraph{暗黑魔法——汇编开栈}~\\
	\lstinputlisting{./source/stack.cpp}
	~\\
	\subsubsection{警告选项}
	建议使用的警告选项:
	\begin{itemize}
		\item \lstinline|-Wall -Wextra|
		\item \lstinline|-Wshadow|:防止局部变量不小心遮盖其他变量
		\item \lstinline|-Wformat=2|:防止 \lstinline|printf|/
		\lstinline|scanf| 写错
		\item \lstinline|-Wconversion|:防止意外的类型转换
		\end{itemize}
	某些情况下有用的警告选项:
	\begin{itemize}
		\item \lstinline|-Wstack-usage=1|:看栈空间使用情况
	\end{itemize}
	~\\
	\subsubsection{运行期检查工具}
	\paragraph{Undefined Behavior Sanitizer:}~\\
	直接在编译时开启编译选项~~ \lstinline|-fsanitize=undefined -g|\\
	\paragraph{Address Sanitizer:}~\\
	直接在编译时开启编译选项~~ \lstinline|-fsanitize=address -g|\\
	\paragraph{Libstdc++ Debug Mode:}~\\
	编译时使用~ \lstinline|-D\_GLIBCXX\_DEBUG|~打开调试模式\\
	此时 libstdc++ 会插入两项运行时检查:迭代器安全性检查 和 算法前提条件检查\\
	~\\
	\subsubsection{性能分析工具}
	\paragraph{gcov/-ftest-coverage -fprofile-arcs:}~\\
		代码覆盖率检测,可以看代码中每一行被执行的次数(精确到行,但只能输出调用次数)\\
	\paragraph{gprof/-pg:}~\\
		代码剖析,可以看函数执行时间占总时间的百分比(输出的是时间,但只能精确到函数)\\
%\end{document}