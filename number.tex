%\documentclass[12pt,a4paper]{article}
\usepackage{xeCJK}
\usepackage{fancyhdr}%页眉页脚
\usepackage{listings}%代码块
\usepackage[margin=1in]{geometry}%页边距
\usepackage{graphics}%图片
\usepackage{fontspec}
\usepackage{amsmath}
\usepackage{amsfonts}
\usepackage{eqnarray}
\usepackage{tabularx}
%定义页眉页脚
\pagestyle{fancy}
\fancyhf{}
\fancyhead[C]{\selectfont \textsc{ACM Template}}
\fancyhead[L]{\rightmark}
\fancyhead[R]{Geometry Rhythm}
%\lfoot{lfoot}
%\cfoot{qkoqhh}
\rfoot{\thepage}
%\renewcommand{\footrulewidth}{0.5pt}

%\rmfamily%罗马字体




\lstset{
	language=C,%代码的语言
	numbers=left,%行号
	frame=single,%
	basicstyle=\footnotesize,
	extendedchars=false,
	basicstyle=\small\ttfamily,
	%	tabsize=2,
	breaklines=true,
	showstringspaces=false
}



\title{ACM Template}
\author{qkoqhh}
%\begin{document}
	\newpage
	\section{数论}
	\subsection{约数个数上界}
	\begin{center}
		\renewcommand\arraystretch{1.5}
		%		\renewcommand\tabcolsep{1pt}
		\begin{tabular}{|l|c|c|c|c|c|c|c|c|c|}
			\hline
			$n\le$ & $10^1$ & $10^2$ & $10^3$& $10^4$& $10^5$& $10^6$& $10^7$ & $10^8$& $10^9$\\
			\hline
			$\max\{w(n)\}$&2&3&4&5&6&7&8&8&9\\
			\hline
			$\max\{d(n)\}$&4&12&32&64&128&240&448&768&1344\\
			\hline
			\hline
			$n\le$ &$10^{10}$& $10^{11}$ & $10^{12}$ & $10^{13}$& $10^{14}$& $10^{15}$& $10^{16}$& $10^{17}$ & $10^{18}$\\
			\hline
			$\max\{w(n)\}$&10&10&11&12&12&13&13&14&15\\
			\hline
			$\max\{d(n)\}$&2304&4032&6720&10752&17280&26880&41472&64512&103680\\
			\hline
		\end{tabular}
	\end{center}
	~\\
	\subsection{EXGCD}
	\lstinputlisting{./source/exgcd.cpp}
	~\\
	\newpage
	
	\subsection{数论小笔记}
	\paragraph{定义一}$(a,b)$ 代表最大公约数,$[a,b]$ 代表最小公倍数
	\paragraph{同余性质的补充}注意一下性质7和性质9即可\\
	性质1(反身性):$a\equiv a\pmod m$\\
	性质2(对称性):若 $a\equiv b\pmod m$, 那么 $b\equiv a\pmod m$\\
	性质3(传递性):若 $a\equiv b\pmod m$,$b\equiv c\pmod m$ ,那么 $a\equiv c\pmod m$\\
	性质4(可加减性):若 $a\equiv b\pmod m$,$c\equiv d\pmod m$,那么 $a\pm c\equiv b\pm d\pmod m$\\
	性质5(可乘性):若 $a\equiv b\pmod m$,$c\equiv d\pmod m$,那么 $ac\equiv bd\pmod m$\\
	证明:设 $a-b=km,c-d=lm$,则 $(ac-bd)=klm^2+(b+d)m$,所以$m|(ac-bd)$\\
	性质6:若 $a\equiv b\pmod m$,那么 $an\equiv bn\pmod m$,其中 $n\in \mathbb{N}$\\
	证明 : $m|(a-b)\to m|n(a-b)$\\
	性质7:若 $ac\equiv bc\pmod m$,$(c,m)=1$,那么 $a\equiv b\pmod m$\\
	证明:$m|c(a-b),d=(m,c)$ => $\frac{m}{d}|(a-b) => a\equiv b\pmod{\frac{m}{d}}$ =>当 $d=1 $时 即 $(c,m)=1$ 上面结论成立\\
	性质8:若 $a\equiv b\pmod m$,那么 $a^n\equiv b^n\pmod m$\\
	证明:$a^n-b^k=(a-b)(a^(n-1)+a^(n-2)b\cdots b^(n-1))$,$m|(a-b)$ ==> $m|(a^n-b^n)$\\
	性质9:若 $a\equiv b\pmod {m_1},a\equiv b\pmod {m_2}\cdots a\equiv b\pmod {m_i}$ 则 $a\equiv b\pmod{[m1,m2,..mi]}$\\
	证明:$m_1 |(a-b),m_2|(a-b) ..m_i|(a-b)$=> $[m1,m2\cdots mi]|(a-b)$\\
	推论:$m_1,m_2..m_i$ 两两互质,则 $a\equiv b\pmod{m_1m_2..m_i}$\\
	\paragraph{定义二}定义在所有正整数上的函数称为算数函数\\
	\paragraph{定义三}算数函数f如果满足对任意两个互素的正整数 $n$ 和 $m$,均有 $f(nm)=f(m)f(n)$,就称为积性函数(乘性函数)。如果对任意两个正整数 $n$ 和 $m$,均有 $f(nm)=f(n)f(m)$,就成为完全积性(乘性)函数。\\
	\textbf{小推论:}对任意积性函数f(n),满足f(1)=1\\
	\paragraph{常见积性函数}~\\
	恒等函数 $I(n)=1$\\
	单位函数 $ id(n)=n$\\
	幂函数 $id^k(n)=n^k$\\
	元函数 $e(n)=[n=1]$      (狄利克雷卷积的乘法单位元)\\
	除数函数 $\sigma _k(n)=\sum_{d|n}d^k$ (表示n的约数的k次幂和)\\
	约束个数函数 $\tau (n)/d(n)=\sigma_0(n)=\sum_{d|n}1$(表示约数个数)\\
	约数和函数 $\sigma(n)=\sigma_1(n)=\sum_{d|n}d$ (表示 $n$ 的约数之和)\\
	欧拉函数 $\varphi(n)$\\
	莫比乌斯函数 $\mu(n)$\\
	\paragraph{定理一}如果 $f$  是一个积性函数,对任意正整数 $n$ 有素数幂分解 $n=p_1^{a_1}p_2^{a_2}\cdots p_k^{a_k}$,那么有 $f(n)=f(p_1^{a_1})f(p_2^{a_2})\cdots f(p_k^{a_k})$\\
	证明:乘积因子之间互质,可以根据积性函数性质分解\\
	\textbf{Ex定理一:}如果 $f$ 是积性函数,则 $f$ 的和函数 $F(n)=\sum_{d|n}f(d)$ 也是积性函数\\
	证明:用狄利克雷卷积易得\\
	\textbf{Ex定理二:}两个积性函数相乘仍为积性函数\\
	证明:令 $h(n)=f(n)g(n)$,$h(nm)=f(nm)g(nm)=f(n)f(m)g(n)g(m)=h(n)h(m)$\\
	\paragraph{定义四}欧拉函数 $\varphi(n)$ 指不超过 $n$ 且于 $n$ 互素的正整数的个数
	\paragraph{定理二}如果 $p$ 是素数,那么 $\varphi(p)=p-1$;反之,如果 $\varphi(p)=p-1$,那么 $p$ 是素数\\
	证明:此命题与 $1..p-1$ 均与素数 $p$ 互质等价
	\paragraph{定理三}设 $p$ 为素数,$a$ 为一个正整数,那么 $\varphi (p^{a})=p^{a}-p^{a-1}=(p-1)*p^{a-1}$\\
	证明:只有 $p$ 的倍数不与 $p^a$ 互质,故在 $p^a$ 基础上减去 $(p^a)/p=p^{a-1}$
	\paragraph{定理四}欧拉函数为积性函数,即 $\varphi (nm)=\varphi(n)\varphi(m)$\\
	证明:
	证明方法很多,下面给出最易懂的证明,因为其他证明窝看不懂orz
	设 $n$ 与 $m$ 互质,构造如下矩阵,包含 $nm$个数
	$$
	\begin{matrix}
	1&2&\cdots&r&\cdots&m\\
	m+1&m+2&\cdots&m+r&\cdots&2m\\
	2m+1&2m+2&\cdots&2m+r&\cdots&3m\\
	\vdots&\vdots&&\vdots&&\vdots\\
	(n-1)m+1&(n-1)m+2&\cdots&(n-1)m+r&\cdots&nm
	\end{matrix}
	$$
	由定义要从矩阵中选出与 $nm$ 互素的数,其行号 $i$ 须满足 $(i,n)=1$,共有 $\varphi(n)$ 个满足条件的 $i$	其列号须满足 $(j,m)=1$,共 $\varphi(m)$ 个满足条件的 $j$\\
	综上共有 $\varphi(n)\varphi(m)$ 个数与 $nm$ 互质,即 $\varphi (nm)=\varphi(n)\varphi(m)$\\
	\paragraph{定理五}设 $n=p_1^{a_1}p_2^{a_2}...p_s^{a_s}$ 为正整数 $n$ 的素数幂分解,那么
	$$
	\varphi(n)=n(1-\frac{1}{p_1})(1-\frac{1}{p_2})...(1-\frac{1}{p_k})
	$$
	证明:结合定理一、定理二、定理四对 $\varphi(n)$ 进行素数幂分解,得
	$$
	\begin{aligned}
	\varphi(n)=&\varphi(p_1^{a_1})\varphi(p_2^{a_2})...\varphi(p_k^{a_k})\\  =&(p_1-1)p_1^{a_1-1}(p_2-1)p_2^{a_2-1}...(p_k-1)p_k^{a_k-1}\\ =&(1-\frac{1}{p_1})p_1^{a_1}(1-\frac{1}{p_2})p_2^{a_2}...(1-\frac{1}{p_k})p_k^{a_k}\\ =&n*(1-\frac{1}{p_1})(1-\frac{1}{p_2})...(1-\frac{1}{p_k})
	\end{aligned}
	$$\\
	\textbf{推论:}当 $n$ 为奇数时,有 $\varphi(2n)=\varphi(n)$\\
	\paragraph{定理六}设 $n$ 是一个大于 $2$ 的正整数,那么 $\varphi(n)$ 是偶数\\
	证明:对任意与 $n$ 互质的数 $m$ ,有 $(n,m)=(n,n-m)=1$。当 $n>2$ 时,$n$ 与 $\frac{n}{2}$ 必不互质,那么有 $m\neq n-m$,即所有与 $n$ 互质的数成对出现,故 $\varphi(n)$ 为偶数\\
	\paragraph{定理七}$\displaystyle \sum_{d|n}\varphi(d)=n$\\
	证明:构造序列 $\displaystyle \frac{1}{n},\frac{2}{n},\frac{3}{n}...\frac{n}{n}$,并进行约分\\
	若分数 $\frac{a}{b}$ 在上面出现,则满足 $b|n$ 和 $(a,b)=1$\\
	对同一个 $b$ ,满足条件的 $a$ 有 $\varphi(b)$个,于是上面 $n$ 个分数根据分母 $b$ 被分为 $\sum_{b|n}\varphi(b) $ 个,得证\\
	\paragraph{定义五}定义小于 $n$ 且和 $n$ 互质的数构成的集合为 $Z_n$ ,称呼这个集合为 $n$ 的完全余数集合。 显然 $\left|Z_n\right|=\varphi(n) $。
	\paragraph{定理八(欧拉定理)}对于互质的正整数 $a$ 和 $n$,$a^{\varphi(n)}\equiv1\pmod n$\\
	证明:令完全余数集合 $Z_n=\left \{x_1,x_2,x_3...x_{\varphi(n)}\right\}$
	\\令$S=\left\{ax_1\pmod n,ax_2\pmod n,....,ax_{\varphi(n)}\pmod n\right\}$ \\
	先证 $Z_n=S$\\
	因为 $a$ 与 $n$ 互质, $x_i(1 \le i \le \varphi(n))$ 与 $n$ 互质, 所以 $a x_i$  与 $n$ 互质,所以 $a  x_i  \bmod n \in Zn$ \\
	若 $i \neq j$ , 那么 $x_i \neq x_j$,有 $0 < x_i - x_j < n$,又 $a$ 与 $n$ 互质,得 $a  ( x_i - x_j ) \bmod n \neq 0$  ,所以 $a  x_i  \neq a  x_j \pmod n$ (消去律)\\
	进而得\\
	$$
	\begin{aligned}
		a^{\varphi(n)}x_1x_2\cdots x_{\varphi(n)}&\equiv~(ax_1)(ax_2)...(ax_{\varphi(n)}) &\pmod n\\ 
		a^{\varphi(n)}x_1x_2\cdots x_{\varphi(n)}& \equiv~ x_1x_2\cdots x_n &\pmod n\\
		a^{\varphi(n)}&\equiv~1 &\pmod n\\
	\end{aligned}
	$$
	\paragraph{推论(费马小定理)}若 $p$ 为质数,对于互质的正整数 $a$ 和 $p$,$a^{p-1}\equiv 1\pmod p$
	除了用欧拉定理推导费马小定理,还有一种直接证明的方法:(课上学的)\\	
	对一个余数 $r$ ,$1\le r\le p-1$,对 $r$ 逐次乘上任意一个数 $a$ 满足 $(a,p)=1$,有 $a^sr\mid p$,由容斥原理,当 $s>n$,必有 $a^sr$与之前的数重复\\
	考虑最早出现重复的情况,若 $a^sr\equiv a^kr\pmod p$,有 $a^{s-k}r\equiv r\pmod p$,因此最早出现重复的情况只能是 $a^{s}r\equiv r\pmod p$\\
	因此根据这个性质我们可以把 $1\cdots p-1$ 分成一个个环,当给 $1\cdots p-1$ 乘 $a$ 模 $p$ 时,会依次得到 $p-1$ 的排列,即
	$$
	(1a)(2a)(3a)..((p-1)a)\equiv (p-1)!\pmod p
	$$
	进而得 $a_{p-1}\equiv1\pmod p$\\
	\paragraph{定理九}当 $n>1$ 时,$1\cdots n$ 中与 $n$ 互质的整数和为 $\frac{n\varphi(n)}{2}$\\
	证明:对任意与 $n$ 互质的数 $m$ ,有 $gcd(n,m)=gcd(n,n-m)=1$,即对任意 $m$ 必有 $n-m$ 与 $n$ 互质,利用倒序相加法可证
	\paragraph{Ex定理一(欧拉降幂)} $a^k\equiv a^{k\bmod \varphi(m)+\varphi(m)}\pmod m$ ($k>\varphi(m)$)\\
	证明:首先,若 $a$ 与 $m$ 互质,由欧拉定理可知 $a^k\equiv a^{k\bmod \varphi(m)}\pmod m$ ,即证上式\\
	若 $(a,m)>1$,证明就比较长了,分为好几步\\
	先证 $(t,a)=1$ ,其中 $t=\frac{m}{(m,a^{\varphi(m)})}$\\
	显然构造 $t$ 是为了排除 $a$ 的因子,而对 $a$ 和 $m$ 的一个公共素因子,有 $v\varphi(m)>=(p_i-1)p_i^(k-1)>k$,所以 $a^{\varphi(m)}$ 足够排除所有公共素因子,所以这 $t$ 和 $a$ 互质
	由于 $t$ 是 $m$ 的因子,所以根据欧拉函数的素因子分解式,存在 $k$ 使 $\varphi(m)=k\varphi(t)$ ,故有
	$$
	a^{\varphi(m)}\equiv a^{k\varphi(t)}\equiv(a^{\varphi(t)})^k\equiv1\pmod t
	$$
	得 $t|(a^{\varphi(m)}-1)$\\
	两边同乘 $(m,a^{\varphi(n)})$\\
	$m|(m,a^{\varphi(m)})(a^{\varphi(m)}-1)$\\
	即 $m|a^{\varphi(m)}(a^{\varphi(m)}-1)$\\
	即 $m|a^{\varphi(m)}(a^{\varphi(m)}-1)$\\
	所以将 $k$ 分解为 $k=p\varphi(m)+q$,其中 $q=k\bmod \varphi(m)$,可得
	$$
	a^k\equiv a^{\varphi(m)+q}\equiv a^{k\bmod \varphi(m)+\varphi(m)}\pmod m
	$$
	~\\
	\paragraph{定义六}因子和函数 $\sigma$ 定义为整数 $n$ 的所有正因子之和,记为 $\sigma(n)$
	\paragraph{定义七}因子个数函数 $\tau$ 定义为正整数 $n$ 的所有正因子个数,记为 $\tau(n)$
	\paragraph{定理十}因子和函数 $\sigma$ 和因子个数函数 $\tau$ 是积性函数\\
	证明:设 $n$ 与 $m$ 互质,$\sigma(n)=k$,$\sigma(m)=s$, $n$ 的因子为 $x_1,x_2,x_3\cdots x_k$,$m$ 的因子为 $y_1,y_2\cdots y_s$\\
	那么 $nm$ 的因子为 $x_1y_1,x_1y_2\cdots x_1y_s,x_2y_1,x_2y_2\cdots,x_2y_s\cdots \cdots x_ky_1,x_ky_2\cdots x_ky_s$ 共有 $ks$ 个因子,所以 $\sigma(nm)=\sigma(n)\sigma(m)$,$\sigma(n)$ 为积性函数\\
	同理
	$$
	\tau(nm)=\sum_{i=1}^{k}\sum_{j=1}^{s}x_iy_j=\sum_{i=1}^{k}x_i\sum_{j=1}^{s}y_j=\tau(n)\tau(m)
	$$
	故 $\tau(n)$ 为积性函数
	\paragraph{定理十一} $\displaystyle \sigma(p^k)=1+p+p^2+..+p^k=\frac{p^{k+1}-1}{p-1}$
	\paragraph{定理十二} $\displaystyle \tau(p^k)=k+1$\\
	~\\
	\paragraph{定义八}莫比乌斯函数
	$$
	\mu(n)=\left\lbrace
	\begin{matrix}
	1&n=1\\
	(-1)^k&n=p_1p_2\cdots p_k\\
	0&othersise\\
	\end{matrix}
	\right.
	$$
	\paragraph{定理十三}莫比乌斯函数为积性函数
	\paragraph{定理十四} $\displaystyle \sum_{d|n}\mu(d)=[n=1]$\\
	证明:由于产生影响的项为 $n=1$ 和不同素数相乘的项,故考虑将 $n$ 的素因子 $p_1,p_2,p3\cdots p_k$ 相互组合形成的数\\
 	则根据二项式定理上式可化为 $\sum_{i=0}^{k}\binom{k}{i}(-1)^i=(1-1)^k$\\
 	等价于 $[n=1]$
 	\paragraph{定理十五(莫比乌斯反演)}~\\
 	形式一:\\
 	若 $\displaystyle F(n)=\sum_{d|n}f(d)$,则有 $\displaystyle f(n)=\sum_{d|n}\mu(d)F(\frac{n}{d})$\\
 	形式二:\\
 	若 $\displaystyle F(n)=\sum_{n|d}f(d)$,则有 $\displaystyle f(n)=\sum_{n|d}\mu(\frac{d}{n})F(d)$\\
 	证明:\\
 	形式一:
 	$$
 	\begin{aligned}
 	\sum_{d|n}\mu(d)F(\frac{n}{d})=&\sum_{d|n}\mu(d)\sum_{k|\frac{n}{d}}f(k)\\=&\sum_{k|n}f(k)\sum_{d|\frac{n}{k}}\mu(d)\\=&\sum_{k|n}f(k)[\frac{n}{k}=1]\\=&f(n)
 	\end{aligned}
 	$$
 	形式二:
 	$$
 	\begin{aligned}
 	\sum_{n|d}\mu(\frac{d}{n})F(d)=&\sum_{k=1}^{+\infty }\mu(k)F(kn)\\=&\sum_{k=1}^{+\infty}\mu(k)\sum_{kn|d}f(d)\\=&\sum_{n|d}f(d)\sum_{k|\frac{d}{n}}\mu(k)\\=&\sum_{n|d}f(d)[\frac{d}{n}=1]=f(n)
 	\end{aligned}
 	$$\\
 	\paragraph{定义九(狄利克雷卷积)}对 $2$ 个算术函数 $f$,$g$,定义其 Dirichlet卷积为新函数 $f*g$ ,满足 $\displaystyle (f*g)(n)=\sum_{d|n}f(d)g(\frac{n}{d})$\\
 	\paragraph{狄利克雷卷积的性质}~\\
 	交换律: $f * g = g * f$\\
 	结合律 :$( f * g ) * h = f * ( g * h )$\\
 	单位元 :$f * e = f$\\
 	分配律: $f * ( g + h ) = f * g + f * h$\\
 	证明:交换律和分配律可根据和式的性质证明,单位元显然成立
 	$$
 	\begin{aligned}
 	(f*g)*h(n)=\sum_{d|n}h(\frac{n}{d})\sum_{k|d} f(k)g(\frac{d}{k})=\sum_{ijk|n}f(i)g(j)h(k)
 	\\
 	f*(g*h)(n)=\sum_{d|n}f(\frac{n}{d})\sum_{k|d} g(k)h(\frac{d}{k})=\sum_{ijk|n}f(i)g(j)h(k)
 	\end{aligned}
 	$$
 	可得结合律成立
 	\paragraph{定理十}两个积性函数的狄利克雷卷积仍为积性函数
 	证明:设 $h = f * g$ ,$n$ 与 $m$ 互质,对 $n$ 和 $m$ 进行素因子分解,有
 	$$
 	\begin{aligned}
 	 	 n=p_1^{a_1}p_2^{a_2}...p_k^{a_k}\\m=q_1^{b_1}q_2^{b_2}...q_s^{b_s}
 	\end{aligned}
 	$$
 	故
 	$$
 	\begin{aligned}
 	h(nm)=&\sum_{d|nm}f(d)g(\frac{nm}{d})\\
 	h(n)=&\sum_{d|n}f(d)g(\frac{n}{d})\\
 	h(m)=&\sum_{d|m}f(d)g(\frac{m}{d})\\
 	\end{aligned}
 	$$
 	对 $h(nm)$ 中的每一项 $d$,对其进行素因子分解,可得 $d=p_1^{c_{11}}p_2^{c_{12}}...p_k^{c_{1k}}q_1^{c_{21}}q_2^{c_{22}}...q_s^{c_{2s}}$\\
 	在 $h(n)$ 和 $h(m)$ 中,仅有一项 $x$ 和 $y$,满足 $d=xy$,且 $x$ 和 $y$ 分别为\\
 	$$
 	\begin{aligned}
 	x=p_1^{c_{11}}p_2^{c_{12}}...p_k^{c_{1k}}\\ 
 	y=q_1^{c_{21}}q_2^{c_{22}}...q_s^{c_{2s}}
 	\end{aligned}
 	$$
 	由此可得,$h(nm)$ 中的每一项均从 $h(n)$ 的对应项和 $h(m)$ 的对应项相乘而成,故 $h(nm)=h(n)h(m)$,得证
 	\paragraph{定理十二} $\displaystyle \sum_{d|n}\frac{n}{d}\mu(d)=\varphi(n)$\\
 	证明:结合 $\sum_{d|n}\varphi(d)=n$ 和莫比乌斯反演定理可得
 	\paragraph{小结}~\\
 	$$
 	\begin{aligned}
 	1*\mu=e\\
 	\varphi *1=id\\
 	\mu *id=\varphi
 	\end{aligned}
 	$$
	\newpage
	\subsection{min25筛}
	这里主要阐述一下算法流程和放一下模板
	\subsubsection{预处理}
	\textbf{\large 准备工作}\\
	
	积性函数中,最重要的是素数,为了求一个积性函数 $f(i)$ 的前缀和,我们先要求 $\sum_p f(p)$ ,即所有素数的函数值的和
	
	这即是 min25 筛的预处理步骤,由于自变量为素数,所以对应的函数形式也比较简单,我们可以以这个函数( min25 筛中要求这个函数为完全积性函数且前缀和比较容易求得 )来进行筛法,将该函数记为 $F(n)$\\
	\paragraph{注意	}\textbf{由于 1 没有最小素因子,min25 筛求解时先暂时忽略 $f(1)$ 的影响}\\
	~\\
	记 $p_i$ 为第 $i$ 小的素数\\
	记 $lpf(n)$ 为 $n$ 的最小素因子\\
	\paragraph{引理}  $\forall i\le n$ ,若 $lpf(i)> \sqrt n$ ,则 $i$ 为质数
	设
	$$
	f(n,j)=\sum_{lpf(i)>p_j||i\in prime}^{i\le n} F(i)
	$$
	可以发现 $f(i,j)$ 是在模拟埃式筛法的过程\\
	有引理可以发现,我们只需要将所有小于等于 $\sqrt n$ 的素数筛完即可\\
	记 $m$ 为最大的满足 $p_m\le\sqrt n$ \\
	并预处理\\
	$$
	pre(n)=\sum_{i=1}^nF(p_j)
	$$
	这里只需要预处理到 $m$ 即可,将在下文筛法中用到\\
	~\\
	~\\
	\textbf{\large 计算}\\
	~\\
	初始状态为
	$$
	f(i,0)=\sum_{j=2}^iF(j)
	$$
	考虑在当前状态中再筛去 $p_j$ ,即筛去 $\forall lpf(i)=p_j$ ,则有
	$$
	f(i,j)=f(i,j-1)-F(p_j)(f(\lfloor\frac{i}{p_j}\rfloor,j-1)-pre(j-1))
	$$
	实现时可以考虑按 $i$ 从大到小更新 $f[]$ 数组\\
	~\\
	\textbf{\large 复杂度分析}\\
	~\\
	可以发现,如果只是求 $f(n,m)$ , $i$ 的有效值个数为 $O(\sqrt n)$ \\
	另外,用 $p_j$ 更新函数值时,若 $i<p_j^2$ , $f(i,j)=f(i,j-1)$ 。即 $p_j$ 只会更新 $i\ge p_j^2$ 的函数值\\
	那么考虑个 $i$ 被更新的次数,其复杂度为
	$$
	\sum_{i=1}^{\sqrt n}\frac{\sqrt i}{\ln \sqrt i}+ \sum_{i=1}^{\sqrt n} \frac{\sqrt \frac{n}{i}}{\ln \sqrt \frac{n}{i}}\le\frac{n^{\frac{3}{4}}}{\ln n}
	$$\\
	~\\
	~\\
	~\\
	\subsubsection{递归求前缀}
	求得素数的函数和之后,就能够为求前缀和带来便利\\
	设
	$$
	S(n,j)=\sum_{lpf(i)\ge p_j}^{i\le n} f(i)
	$$
	和 $f(n,m)$ 的求解相反,我们需要求出 $S(n,1)$\\
	依次枚举最小素因子,可以得到
	$$
	S(n,j)=f(n,m)-pre(j-1)+\sum_{k=j}^m(\sum_e^{p_k^e\le n}f(p_j^e)(S(\lfloor\frac{n}{p_j^e}\rfloor,k+1)+\sum_{e=2}^{p_j^e\le n}f(p_k^e))
	$$
	这里直接递归即可,不用记忆化\\
	~\\
	\textbf{\large 复杂度分析}\\
	还不会\\
	~\\
	~\\
	~\\
	代码见下页
	\newpage
	\lstinputlisting{./source/min25.cpp}
	\newpage
	\subsection{二次剩余}
	\paragraph{题意}~\\
	求解 $x^2\equiv t\pmod p$\\
	\paragraph{解法}~\\
	欧拉准则:\\
	令勒让德符号 $\left(\frac{t}{p} \right)=t^{\frac{p-1}{2}}\bmod p$\\
	当 $\left(\frac{t}{p}\right)=0$ 时, $p|t$\\
	当 $\left(\frac{t}{p}\right)=1$ 时, $t$ 为二次剩余\\
	当 $\left(\frac{t}{p}\right)=-1$ 时, $t$ 为非二次剩余\\
	证明:\\
	$t$ 与 $p$ 不互质显然$\left(\frac{t}{p}\right)=0$\\
	当 $t$ 与 $p$ 互质\\
	$$
	t^{p-1}\equiv1\pmod{p}
	$$
	所以\\
	$$
	t^{\frac{p-1}{2}}\equiv\pm1\pmod p
	$$
	必要性:\\
	当 $t$ 为二次剩余,那么存在 $x$ ,满足\\
	$$
	\begin{aligned}
	x^2&\equiv&t&\pmod p\\
	x^{p-1}&\equiv&t^{\frac{p-1}{2}}&\pmod p\\
	t^{\frac{p-1}{2}}&\equiv&1&\pmod p
	\end{aligned}
	$$
	充分性:\\
	令 $g$ 为 $p$ 的一个原根,且 $t=g^k$ ,那么\\
	$$
	g^{\frac{k(p-1)}{2}}\equiv 1 \pmod p
	$$
	由于 $g$ 的指标是 $p-1$ ,所以 $\frac{k(p-1)}{2}|(p-1)$ ,即 $\frac{k}{2}|1$ \\
	所以 $k|2$ ,$t$ 为一个二次剩余\\
	那么 $-1$ 的情况也一起证明了\\
	~\\
	有这个结论就可以用于构造了,任取一个数 $a$ ,满足 $w=a^2-t$ 为一个非二次剩余,那么 $x=(a+\sqrt w)^{\frac{p+1}{2}}$ 为一个二次剩余方程的解\\
	证明:\\
	首先证明 $(a+b)^p\equiv a^p+b^p\pmod p$\\
	这个其实模方程的一个结论,因为展开之后会有二项式系数 $\binom{p}{i}$ ,由于 $p$ 为素数,所以只有第一项和最后一项不会被 $p$ 整除\\
	然后有
	$$
	\begin{aligned}
	(a+\sqrt w)^{p+1}&=(a+\sqrt w)(a^p+{\sqrt w}^p)
	\\&=(a+\sqrt w)(a+w^{\frac{p-1}{2}}\sqrt w)
	\\&=(a+\sqrt w)(a-\sqrt w)
	\\&=a^2-w
	\\&=n\pmod p
	\end{aligned}
	$$
	可是既然 $w$ 是非二次剩余,那又如何用 $\sqrt w$ 计算呢?把他们都看成 $x+y\sqrt w$ 的形式,不管这个数做何种运算,都会像复数一样最终化成 $x+y\sqrt w$ 的形式,因此只要算系数 $x$ 和 $y$ 就可以了,然后直接做快速幂即可。。\\
	~\\
	~\\
	\paragraph{代码实现}~\\
	\lstinputlisting{./source/sqrt.cpp}
	
	\newpage
	\subsection{BSGS}
	这个板子需要特判 $a=0$ 的情况\\
	\lstinputlisting{./source/bsgs.cpp}
	\newpage
	\subsection{EXBSGS}
	EXBSGS 就是在 BSGS 的基础上疯狂约去 GCD\\
	\lstinputlisting{./source/exbsgs.cpp}
	\newpage
	\subsection{N次剩余/原根}
	求解 $x^a\equiv b(mod\,p)$\\
	由于 $p$ 为质数,令 $g$ 为 $p$ 的一个原根\\
	那么 $a\,ind_gx\equiv ind_gb(mod\,\,\varphi(p))$\\
	原根的求解可以依次枚举 $g$ ,检验 $g^d\equiv 1(mod\,p)$ ,其中 $d<\varphi(p)$\\
	如果这个 $d$ 存在,那么 $d$ 的倍数也必然成立,那么只要取 $d=\frac{\varphi(p)}{p_i}$ 检验即可,其中 $p_i$ 为$\varphi(p)$ 的素因子\\
	由于原根的大小一般在 $O(n^{0.25})$ 以内,所以求原根的复杂度为 $O(n^{0.25}logp\times loglogp)$\\
	求出 $g$ 之后利用 $BSGS$ 求出 $ind_gb$ 进而求出 $ind_gx$ 即可\\
	\lstinputlisting{./source/nsqrt.cpp}
	\newpage
	\subsection{奇模数N次剩余}
	模数是奇数应该是为了保证原根的存在。。\\
	设原方程解为 $X$\\
	然后由于模数可以为合数所以要用 $CRT$ 分解,类似\\
	$$
	x^a\equiv b(mod\,\,p^k)
	$$
	设其解为 $w_1,w_2...w_j$ ,那么
	$$
	X\equiv w_j(mod\,\,p^k)
	$$
	必有且仅有一个成立\\
	这样可以组成若干个方程组,根据 $CRT$  一个方程组在模数范围内仅有一解,那么解的个数即为方程组的个数,所以只需要计算每个方程组的解集大小,直接相乘即可。。\\
	回到
	$$
	x^a\equiv b(mod\,\,p^k)
	$$
	此时模数不是质数,不能直接求对数\\
	当 $(b,p^k)=1$ ,此时可以直接求对数后求解,若存在解,那么解个数为 $gcd((p-1)p^{k-1},a)$\\
	当 $b\bmod p^k=0$ ,那么 $x=Cp^t$ ,那么 $at\geq k$ ,所以 $t=\left\lceil \frac{k}{a} \right\rceil$\\
	那么解个数为
	$$
	\frac{p^k}{p^t}=p^{k-\left\lceil \frac{k}{a} \right\rceil}
	$$
	当 $1<(b,p^k)<p^k$ ,令 $b=Bp^{cnt}$ ,那么
	$$
	\left(\frac{x}{p^{\frac{cnt}{a} }}  \right)^a\equiv B(mod\,\,p^{k-cnt})
	$$
	此时满足 $a|cnt$ 才有解,然后就可以求离散对数了。。\\
	但是解集的范围均落在 $[0,p^{k-cnt})$ 之间,那么需要将解集扩大到 $[0,p^{k-\frac{cnt}{a}})$ ,所以方案数要乘上 $p^{cnt-\frac{cnt}{a}}$\\
	相当考验板子。。\\
	\lstinputlisting{./source/nsqrt2.cpp}
%	\newpage
%	\subsection{}
%	\lstinputlisting{./source/}
%\end{document}