%\documentclass[12pt,a4paper]{article}
\usepackage{xeCJK}
\usepackage{fancyhdr}%页眉页脚
\usepackage{listings}%代码块
\usepackage[margin=1in]{geometry}%页边距
\usepackage{graphics}%图片
\usepackage{fontspec}
\usepackage{amsmath}
\usepackage{amsfonts}
\usepackage{eqnarray}
\usepackage{tabularx}
%定义页眉页脚
\pagestyle{fancy}
\fancyhf{}
\fancyhead[C]{\selectfont \textsc{ACM Template}}
\fancyhead[L]{\rightmark}
\fancyhead[R]{Geometry Rhythm}
%\lfoot{lfoot}
%\cfoot{qkoqhh}
\rfoot{\thepage}
%\renewcommand{\footrulewidth}{0.5pt}

%\rmfamily%罗马字体




\lstset{
	language=C,%代码的语言
	numbers=left,%行号
	frame=single,%
	basicstyle=\footnotesize,
	extendedchars=false,
	basicstyle=\small\ttfamily,
	%	tabsize=2,
	breaklines=true,
	showstringspaces=false
}



\title{ACM Template}
\author{qkoqhh}
%\begin{document}
	\newpage
	\section{动态规划}
	\subsection{数位DP}
	\lstinputlisting{./source/digit.cpp}
	\newpage
	\subsection{斜率优化}
	这里仅提供写代码的样式
	\vspace{0.5cm}
	\lstinputlisting{./source/slope.cpp}
	\newpage
	\subsection{单调队列优化背包}
	设 $d[i][j]$ 为前 $i$ 种物品体积为 $j$ 的最大价值\\
	$$
	d[i][j]=\max\{d[i-1][j-kv_i]+kw_i\}
	$$
	此时$0\leq k\leq \min\{\frac{j}{v_i},c[i]\}$\\
	\\
	令 $j=pv_i+q$,$k=p-k$,有\\
	$$
	d[i][j]=\max\{d[i-1][q+(p-k)v_i]+kw_i\}=\max\{d[i-1][q+kv_i]-kw_i\}+pw_i
	$$
	此时 $p-\min\{\frac{j}{v_i},c[i]\}\leq k\leq p$ ,即 $\max\{0,p-c[i]\}\leq k\leq p$\\
	\\
	然后考虑到枚举 $k$ 的时候 $k$ 一定为整数,所以直接 $p-c[i]\leq k\leq p $ 就行,然后发现这就是划窗\\
	所以枚举 $q$ 后枚举 $k$,然后直接单调队列维护最值即可\\
	\\
	然后时间复杂度降成了 $O(nm)$ ,而且好像并不比二进制压缩复杂多少,所以以后就写单调队列了\\
	\lstinputlisting{./source/bag.cpp}
	\newpage
	\subsection{悬线法}
	如果找不含 $0$ 的最大子矩阵,这个子矩阵必然满足一个性质,就是这个子矩阵的边缘一定顶着 $0$ 或者是边界,根据这个性质可以降低计数的复杂度,先预处理三个数组 $up$ 和 $left$ 和 $right$,分别代表 $(i,j)$ 往上拓展的长度,还有往左拓展的边界和往右拓展的边界\\
	然后再进行如下处理\\
	若 $(i,j)$ 和 $(i,j-1)$ 可以为合法子矩阵的元素\\
	$$
	\begin{aligned}
	left[i][j]&=&\min&\{left[i][j],left[i-1][j]\}\\
	right[i][j]&=&\min&\{right[i][j],right[i-1][j]\}
	\end{aligned}
	$$\\
	这样就能处理出以 $x=i$ 为底边,$(i,j)$ 为基点,拓展出去的子矩阵的边界\\
	\lstinputlisting{./source/hangup.cpp}
	\newpage
	\subsection{决策单调性}
	这里仅给出样例
	\vspace{0.5cm}
	\paragraph{题意}给定 $n$ 个数的序列,要求将序列分成 $m$ 段,使得这 $m$ 段的方差最小,输出方差乘上 $m^2$
	\subsubsection{解法一:二分}
	要保证整数,直接把 $m^2$ 乘积方差的和式里面变成 $(mx_i-sum_x)^2$,然后求这个的最小值就可以了\\
	设 $a$ 为前缀和,设 $d[i][j]$ 为第 $i$ 段以 $j$ 位置为结尾的最小值\\
	$$
	d[i][j]=\max\{d[i-1][k]+(a[j]-a[k]-sum_x)^2\}
	$$
	而 $f(x)=(x-C)^2$ 是个下凸函数,即 $x$ 越大增长速度越快\\
	考虑两个决策点 $k<v<i$,如果\\
	$$
	d[i-1][k]+(a[j]-a[k]-sum_x)^2<d[i-1][v]+(a[j]-a[v]-sum_x)^2
	$$
	那么此后随着 $j$ 增加,此后 $k$ 不会比 $v$ 更优\\
	所以用决策单调性存递增的决策点,用上面的式子排除对头即可\\
	\lstinputlisting{./source/opt1.cpp}
	\newpage
	\subsubsection{解法二:分治}把方差中的平方项
	拆出来化为 $m\sum x_i^2-sum_x^2$\\
	主要就是 $[x,y]$ 是当前的维护区间, $[l,r]$ 是决策区间,然后暴力求出 $mid$ 的决策点 $pos$,又根据决策单调性,$[x,mid-1]$ 的决策点必在 $[l,pos]$ 上,$[mid+1,y]$ 的决策点必在 $[pos,r]$ 上,然后一直分治下去就行了\\
	\lstinputlisting{./source/opt2.cpp}
	\vspace{0.5cm}
	\subsubsection{解法三:斜率优化}~\\此处省略
	\newpage
	\subsection{DP凸优化}
	\vspace{0.5cm}
	这里只给出例题\\
	\paragraph{题意}有 $n$ 户人家,要建 $m$ 个体育馆,要安排这 $m$ 个体育馆,使得每户到最近的体育馆的距离之和最小\\
	大体是对建一个体育馆,即支配一个区间需要付出额外的代价 $t$,这样只要二分 $t$ 就能使最优决策时的区间数变成 $m$,从而直接无视 $m$ 的限制,直接做单调 $DP$\\
	设 $d[i]$ 为到 $i$ 的代价\\
	$$
	d[i]=\max\{d[j]+cost(j+1,i)+t\}
	$$
	\\其中 $cost$ 函数满足四边形不等式,即有凸函数的性质,所以可以利用决策单调性二分决策点,然后优化成 $O(n\log n)$,总复杂度为 $O(nlognlogC)$ ,$C$ 为二分范围\\
	感觉边界比较难处理,想用浮点数去二分的,然而会TLE\\
	这个问题的解决方案在wqs的《浅析一类二分方法》一文中有提到,并指出直接用整数二分寻找分段数>=k的答案即可\\
	\\
	代码见下页\\
	\newpage
	\lstinputlisting{./source/wqs.cpp}
	\newpage
	\subsection{斯坦纳树}
	\subsubsection{概念}
	斯坦纳树问题是组合优化学科中的一个问题。将指定点集合中的所有点连通,且边权总和最小的生成树称为最小斯坦纳树(Minimal Steiner Tree),其实最小生成树是最小斯坦纳树的一种特殊情况。而斯坦纳树可以理解为使得指定集合中的点连通的树,但不一定最小。\\
	求解斯坦纳树是NP问题,所以要关键点的个数要少(一般 $m\le 10$ ),然后才可以状压\\
	\subsubsection{求解方法}
	令 $d[i][j]$ 为覆盖了 $i$ 点集,$j$ 为根的树的最小权值\\
	那么状态转移有两重:
	\paragraph{第一重}:把两个子状态合并成一个状态
	$$
	d[i][j]=\max\limits_{t\subseteq i} (d[t][j]+d[i-t][j])
	$$
	复杂度 $O(V3^m)$\\
	枚举子状态的时候可以参考挑战的位运算枚举方法\\
	\paragraph{第二重}:在同一状态中进行转移(就是 $SPFA$ )\\
	$$
	d[state][j]=max(d[state]][j],d[state][i]+e[i][j])
	$$
	复杂度 $O(E2^m)$\\
	\vspace{5.7cm}
	\lstinputlisting{./source/mst.cpp}
%	\newpage
%	\subsection{}
%	\lstinputlisting{./source/}
%\end{document}