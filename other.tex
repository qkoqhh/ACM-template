%\documentclass[12pt,a4paper]{article}
\usepackage{xeCJK}
\usepackage{fancyhdr}%页眉页脚
\usepackage{listings}%代码块
\usepackage[margin=1in]{geometry}%页边距
\usepackage{graphics}%图片
\usepackage{fontspec}
\usepackage{amsmath}
\usepackage{amsfonts}
\usepackage{eqnarray}
\usepackage{tabularx}
%定义页眉页脚
\pagestyle{fancy}
\fancyhf{}
\fancyhead[C]{\selectfont \textsc{ACM Template}}
\fancyhead[L]{\rightmark}
\fancyhead[R]{Geometry Rhythm}
%\lfoot{lfoot}
%\cfoot{qkoqhh}
\rfoot{\thepage}
%\renewcommand{\footrulewidth}{0.5pt}

%\rmfamily%罗马字体




\lstset{
	language=C,%代码的语言
	numbers=left,%行号
	frame=single,%
	basicstyle=\footnotesize,
	extendedchars=false,
	basicstyle=\small\ttfamily,
	%	tabsize=2,
	breaklines=true,
	showstringspaces=false
}



\title{ACM Template}
\author{qkoqhh}
%\begin{document}
	\newpage
	\section{Other}
	\subsection{矩阵树定理}
	矩阵树定理用于生成树计数,构造基尔霍夫矩阵如下:
	\begin{itemize}
		\item $a[i][i]$ 为点 $i$ 的度数
		\item $a[i][j]=a[j][i]$ 为点 $i$ 和 $j$ 之间的边数的相反数
	\end{itemize}
	然后求解该矩阵的 $n-1$ 阶子式即可,可直接求前 $(n-1)\times(n-1)$ 矩阵的行列式\\
	\vspace{1cm}
	~\\
	在有向图中,若需要外向树计数,可构造基尔霍夫矩阵如下:
	\begin{itemize}
		\item $a[i][i]$ 为点 $i$ 的入度
		\item $a[i][j]$ 为点 $i$ 到 $j$ 的边数的相反数
	\end{itemize}
	同样求解 $n-1$ 阶子式
	\vspace{1cm}
	~\\
	附求解行列式的代码
	\lstinputlisting{./source/matrix.cpp}
	
	\newpage
	\subsection{蒙哥马利算法}
	这个算法其实是用来解决RSA中计算公钥的问题的,是个底层常数优化算法。。\\
	主要解决两个问题:
	\begin{itemize}
		\item 蒙哥马利约减,即 $tp^{-1} \pmod m$
		\item 蒙哥马利乘模,即 $xy \pmod m$
	\end{itemize}
	这个算法的目的就是避免取模和除法来降低常数\\
	算法过程,直接看杜教板子比看网上的博客来得简单。。\\
	\subsubsection{预处理参数}
	$b$ 为数制,这里所有的数都以 $b$ 进制的蒙哥马利表示法表示\\
	$$
	\begin{aligned}
	p=b^k (p\ge m)\\
	inv=m^{-1}\ \bmod\ p\\
	r_2=p^2\ \bmod\ m\\
	\end{aligned}	
	$$
	取 $b$ 进制是为了方便进行位运算,所以 $b$ 一般取 $2$,然后 $p$ 取 $2^{64}$\\
	$2^{64}$ 的话取模的时候自然溢出就行了,比较方便\\
	然后关于蒙哥马利表示法,就是 $x$ 的蒙哥马利表示为 $xp \bmod m$\\
	所有数的运算都在此表示法下进行\\
	\subsubsection{蒙哥马利约减}
	若$t\le m^2$,可以用$(t-(t\times inv \bmod p)\times m )/p$  代替  $tp^{-1} \bmod  m$\\
	证明:\\
	因为
	$$
	t-t\times inv\times m\equiv t+t(-m^{-1}\times m)\equiv0\pmod p
	$$
	所以
	$$
	p|(t+t\times inv\times m)
	$$
	即
	$$
	p|(t+(t\times inv\bmod p)\times m)
	$$
	故有
	$$
	(t+(t\times inv\bmod p)\times m )/p\equiv tp^{-1}+(t\times inv\bmod p)p^{-1}m\equiv tp^{-1}\pmod m
	$$
	在数值上,由于\\
	$$
	\begin{aligned}
	(t-(t\times inv\ mod \ p\times m)/p)\le m^2/p\le m\\
	(t-(t\times inv\ mod \ p\times m)/p)\ge -m
	\end{aligned}
	$$
	所以只需判断正负即可\\
	然后就可以用蒙哥马利约减来求出一个数的蒙哥马利表示了,直接对$x\times r_2$进行约减即可\\
	从蒙哥马利表示法中还原出原数也只需做一个约减就可以了\\
	\subsubsection{蒙哥马利乘模}
	令
	$$
	\begin{aligned}
	\hat{x}=xp \bmod m\\
	\hat{y}=yp \bmod m
	\end{aligned}
	$$
	那么在蒙哥马利表示法中,$\hat{xy}=xyp$\\
	即$\hat{xy} \ \equiv \ \hat{x}\hat{y}/p\pmod m$\\
	直接对$x\times y$进行一次约减就可以了\\
	\vspace{1.7cm}~\\
	基本就这么多。。说那么多其实只要有板子就可以。。
	\lstinputlisting{./source/mege.cpp}
%	\newpage
%	\subsection{行列式}
%	\lstinputlisting{./source/}
	\newpage
	\subsection{gcc内建二进制函数}
	\begin{itemize}
		\item \lstinline|__builtin_popcount(x)| 表示 $x$ 二进制中 $1$ 的个数。
		\item \lstinline|__builtin_parity(x)| 表示 $x$ 二进制中 $1$ 的个数的奇偶性。
		\item \lstinline|__builtin_clz(x)| 表示 $x$ 二进制中前导 $0$ 的个数。
		\item \lstinline|__builtin_ctz(x)| 表示 $x$ 二进制中结尾 $0$ 的个数。
		\item \lstinline|__builtin_ffs(x)| 表示 $x$ 二进制中右起第一个 $1$ 的位置。
	\end{itemize}
	以上函数传入的参数为 unsigned int,如需对 unsigned long long 使用,需要在函数名的最后加上 ll,如 \lstinline|__builtin_popcountll|。\lstinline|__builtin_clz(x)| 和 \lstinline|__builtin_ctz(x)| 两者在 $x=0$ 时未定义
	\newpage
	\subsection{高精度}
	不造从哪扒来的高精度板子,功能比较完整,只是不支持负数\\
	\lstinputlisting{./source/biginteger.cpp}
	\newpage
	\subsection{python}
	这里整合了我写过的所有的Python代码,少的可怜
	\paragraph{Code1}~\\
	\lstinputlisting[language=python]{./source/code3.py}
	\newpage
	\paragraph{Code2}~\\
	\lstinputlisting[language=python]{./source/code1.py}
	\newpage
	\paragraph{Code3}~\\
	\lstinputlisting[language=python]{./source/code2.py}
%	\newpage
%	\subsection{}
%	\lstinputlisting{./source/}
%\end{document}